\documentclass{article}
\usepackage[utf8]{inputenc}
\usepackage{mathtools}
\usepackage{listings}

\usepackage{color}
\usepackage{graphicx}
\usepackage{amsmath,amsthm,amssymb,amsfonts}
\usepackage[spanish]{babel}
\usepackage{algorithm}
%\usepackage{arevmath}     % For math symbols
\usepackage[noend]{algpseudocode}

\usepackage{natbib}
\usepackage{graphicx}
\definecolor{mGreen}{rgb}{0,0.6,0}
\definecolor{mGray}{rgb}{0.5,0.5,0.5}
\definecolor{mPurple}{rgb}{0.58,0,0.82}
\definecolor{backgroundColour}{rgb}{0.95,0.95,0.92}


\lstset{
  frame=tb,
  language=C++,
  aboveskip=3mm,
  belowskip=3mm,
  showstringspaces=false,
  columns=flexible,
  basicstyle={\small\ttfamily},
  numbers=none,
  numberstyle=\tiny\color{gray},
  keywordstyle=\color{blue},
  commentstyle=\color{dkgreen},
  stringstyle=\color{mauve},
  breaklines=true,
  breakatwhitespace=true,
  tabsize=3
}

\definecolor{dkgreen}{rgb}{0,0.6,0}
\definecolor{gray}{rgb}{0.5,0.5,0.5}
\definecolor{mauve}{rgb}{0.58,0,0.82}

\DeclarePairedDelimiter\ceil{\lceil}{\rceil}
\DeclarePairedDelimiter\floor{\lfloor}{\rfloor}

\title{ \textbf{Proyecto ADA} }

\author{
Cesar Madera$^{1}$, Enrique Sobrados$^{2}$, Johan Tanta$^{3}$\\ \\
\small{$^{1-3}$Ciencia de la Computación}\\
\small{Universidad de Ingeniería y Tecnología, Barranco}\\
\small{\texttt{\{$^{1}$cesar.madera,$^{2}$enrique.sobrados,$^{3}$johan.tanta\}@utec.edu.pe } }
}
\date{\small{Junio 16, 2020}}


\begin{document}
\maketitle
\tableofcontents
\newpage


\section{Algoritmo Voraz}
\subsection{Eleccion voraz}
%Para el algoritmos voraz ya que nuestro limite de tiempo es O(max{m,n}) nuestro algoritmo se basa en verificar para un par i,j de la lista de bloques A y B respectivamente; si es que el bloque j es mayor que el bloque j+1, si es asi entonces tomaremos el par i,j y avanzaremos el puntero j. Si esto no se cumple hacemos una verificación para i si es que es menor que i+1 entonces tomamos la tupla i,j y avanzamos el puntero en i. Por último si no cumple ninguna de estas condiciones también tomamos la tupla y avanzamos ambos punteros. Estas tomas seran siempre y cuando no rompan con las condiciones de matching por lo que se pusieron algunas variables para controlar esto. Una vez lista los matchings se usara una función para calcular el peso.
%
Sea i, j punteros en los vectores de los bloques A y B respectivamente que empiezan desde el inicio. \textbf{Eleccion Voraz}: 

\begin{itemize}
\item Empezar a realizar un subconjunto con índice i, un $i-división$ con la condición: $B_{j} > B_{j+1}$
\item Al no cumplir la anterior condición, empezar a realizar un subconjunto con índice j, un $j-agrupamiento$ con la condición:
$A_{i} < A_{i+1}$
\end{itemize}

\subsection{Pseudocódigo}

\begin{algorithm}
\caption{Devuelve el Peso}
\begin{algorithmic}[1]
\Procedure{Weight}{matchings}
    \State{suma\_total = 0}
    \State{tempa = A[matchings[1].first].longitud}
    \State{tempb = B[matchings[1].second].longitud}
    \State{t\_a = matchings[0].first}
    \State{t\_b = matchings[0].second}
    \For{i = 2 \textbf{to} size(matchings)}
        \If{matchings[i].first == t\_a}
            \State{tempb += B[matchings[i].second].longitud}
        \ElsIf{matchings[i].second == t\_b}
            \State{tempa += A[matchings[i].first].longitud}
        \Else
            \State{suma\_total += tempa/tempb}
            \State{t\_a = matchings[i].first}
            \State{t\_b = matchings[i].second}
            \State{tempa = A[matchings[i].first].longitud}
            \State{tempb = B[matchings[i].second].longitud}
        \EndIf
    \EndFor
    \State{suma\_total += tempa/tempb}
    \State{\Return{suma\_total}}
\EndProcedure
\end{algorithmic}
\end{algorithm}


\begin{algorithm}
\caption{Devuelve un Match entre A y B}
\begin{algorithmic}[1]
\Procedure{Greedy\_min}{$A,B$}
    \State{i = 1}
    \State{j = 1}
    \State{cont\_B = 0}
    \State{cont\_A = 0}
    \While{ i $<$ size(A) -1  \textbf{and} j $<$ size(B) - 1}
        \If{B[j + 1].longitud $<$ B[j].longitud}
            \If{cont\_B}
                \State{j++}
                \State{i++}
                \State{cont\_B = 0}
            \Else
                \State{matchings.push(i,j)}
                \State{j++}
                \State{cont\_A++}
            \EndIf
        \ElsIf{A[i].longitud $<$ A[i + 1].longitud}
            \State{matchings.push(i,j)}
            \If{cont\_A}
                \State{i++}
                \State{j++}
                \State{cont\_A = 0}
                \State{cont\_B = 0}
            \Else
                \State{i++}
                \State{cont\_B = 0}
            \EndIf
        \Else
            \State{matchings.push(i, j));}
            \State{i++}
            \State{j++}
            \State{cont\_B = 0}
        \EndIf
    \EndWhile
    \If{i == size(A) - 1} 
        \While{j $<$ size(B)} 
            \State{matchings.push(i, j)}
            \State{j++}
        \EndWhile
    \Else 
        \While{i $<$ size(A)}
            \State{matchings.push(i, j)}
            \State{i++}
        \EndWhile
    \EndIf
    \State{\Return matchings, Weight(matchings) }
\EndProcedure
\end{algorithmic}
\end{algorithm}
\newpage
\subsection{Tiempo de ejecucion}
El tiempo de ejecución para este algoritmo en el peor de los casos es: 
    La linea 6 se ejecuta m+n-1 veces sin contar las constantes y en la linea 31 se ejecutaa ese 1 faltante. La funcion Weight tiene un tiempo de ejecucion de max\{m.n\} ya que va a iterar en el maximo número de tuplas de matchings el cual es max\{m,n\}. 
    Por lo tanto el tiempo de ejecucion del algoritmo: $T(m,n) = m+n+max\{m,n\}$ \\ \\
\subsection{Demostración}
\textbf{Demostrar: $T(m,n) = O(max\{m,n\})$}

\begin{eqnarray*}
m+n & \leq & 2 \times max\{m,n\} \\
m+n+ max\{m,n\}  & \leq & 2 \times max\{m,n\} + max\{m,n\} \\
m+n+ max\{m,n\}  & \leq & 3 \times max\{m,n\}, C_1 = 3 \\
T(m,n)& =&  O(max\{m,n\}) 
\end{eqnarray*}
\section{Opt($i,j$)}
\subsection{Solución óptima}
Sea X la solución óptima del problema. Asimismo, sea P y Q subconjuntos.
\begin{itemize}
\item Donde P y el índice $j$, es un $j-agrupamiento$ definido:
\begin{eqnarray*}
P &=& (k,j),(k+1,j), \dots ,(i-1,j),(i,j) \hspace{1cm} 2 \leq k \leq i
\end{eqnarray*}
\item Donde Q y el índice $i$, es una $i-division$ definida:
\begin{eqnarray*}
Q &=& (i,l),(i,l+1), \dots ,(i,j-1),(i,j) \hspace{1cm} 2 \leq l \leq j
\end{eqnarray*}
\end{itemize}

X debe incluir una solución óptima entre los subconjuntos P y Q, debido a ello se observan los siguientes escenarios:

\begin{itemize}
\item Si $ P \in X $:  Luego X debe incluir una solución óptima del subproblema que está dado por los bloques de $A = \{ A_1, A_2,\cdots,A_{k-1} \}$ y de $B = \{B_1,B_2,\cdots,B_{j-1}\}$.
\item Si $ Q \in X $:  Luego X debe incluir una solución óptima del subproblema que está dado por los bloques de $A = \{ A_1, A_2,\cdots,A_{i-1} \}$ y de $B = \{B_1,B_2,\cdots,B_{l-1} \}$.
\end{itemize}

\subsection{Planteamiento de la Recurrencia}
Para calcular el peso asociado a una agrupación es calculada por:
\begin{eqnarray*}
W_A(r,s,j) & = & \frac{A_r+A_{r+1}+\dots+A_s}{B_j}
\end{eqnarray*}

Para calcular el peso asociado a una división es calculada por:
\begin{eqnarray*}
W_D(i,m,n) & = & \frac{A_i}{B_m+B_{m+1}+\dots+B_n}
\end{eqnarray*}

Asimismo, para cada $(i,j)$ se define:
\begin{eqnarray*}
M_A(i,j) & = & min_{k=i}^{2}\left( W_A(k,i,j) + Opt(k-1,j-1) \right) \\ 
M_D(i,j) & = & min_{l=j}^{2}\left( W_D(i,l,j) + Opt(i-1,l-1) \right)
\end{eqnarray*}

Se plantea la siguiente recurrencia
\begin{equation*}
Opt(i,j) =
\begin{cases}
W_A(1,i,1) & \text{j == 1}\\
W_D(1,1,j) & \text{i == 1}\\
min( M_A(i,j) , M_D(i,j) ) &  \text{Caso contrario}
\end{cases}
\end{equation*}

\section{Algoritmo Recursivo}
\subsection{Pseudocódigo}
Se definen las siguientes funciones:
\begin{algorithm}
\caption{Devuelve las tuplas y el peso de un i-division}
\begin{algorithmic}[1]
\Procedure{MatchDivision}{$i,m,n$}
    \State{Tuplas MatchD}
    \For{p = m \textbf{to} n}
        \State{MatchD.push( i , p )}
    \EndFor
    \Return{ MatchD }
\EndProcedure
\end{algorithmic}
\end{algorithm}

\begin{algorithm}
\caption{Devuelve las tuplas y el peso de un j-agrupamiento}
\begin{algorithmic}[1]
\Procedure{MatchGroup}{$r,s,j$}
    \State{Tuplas MatchG}
    \For{p = r \textbf{to} s}
        \State{MatchG.push( p , j )}
    \EndFor
    \Return{ MatchG }
\EndProcedure
\end{algorithmic}
\end{algorithm}

Luego se define la función Opt(i,j):
\newpage
\begin{algorithm}
\caption{Devuelve el min matching}
\begin{algorithmic}[1]
\Procedure{Opt}{$i,j$}       
    \If{$i== 1$}
    \State \Return MatchDivision(i, 0, j)
    
    \ElsIf{$j==1$}
    \State  \Return MatchGroup(0, i, j)
    
    \Else
    \State{Tuplas min\_resultk.weight = $\infty$}
    \State{Tuplas min\_resultl.weight = $\infty$}
    \For{k = i \textbf{down to} 2}
    \State{Match = MATCHGROUP( $k, i, j$ )}
    \State{SubProblem = Opt( $k-1, j-1$ )}
    \State{result = SubProblem + Match }
    \If{min\_resultk.weight $>$ result.weight}
    \State{min\_resultk = result}
    \EndIf
    \EndFor
    
    \For{l = j \textbf{down to} 2}
    \State{Match = MATCHDIVISION( $i, l, j$ )}
    \State{SubProblem = Opt( $i-1, l-1$ )}
    \State{result = SubProblem + Match }
    \If{min\_resultl.weight $>$ result.weight}
    \State{min\_resultl = result}
    \EndIf
    \EndFor
    \State{\Return min( min\_resultl, min\_resultk )}
    \EndIf
\EndProcedure

\end{algorithmic}
\end{algorithm}

\subsection{Tiempo de Ejecución}
Al analizar el tiempo de ejecución del Algorithm 5, se obtiene lo siguiente: 

\begin{equation*}
Opt(i,j) =
\begin{cases}
c  & \text{i == 1 $\vee$  j == 1}\\
T(i,j) = \sum^2_{k=i}T(k-1,j-1)+\sum^{2}_{k=j}T(i-1,k-1) & \text{Caso contrario}
\end{cases}
\end{equation*}

Probaremos por induccion que $T(i,j)=\Omega(2^{max(i,j)})$, con $c = \frac{1}{2}$:

\begin{itemize}
\item Como caso base, donde i = 1, j = 1 y $c_1 = 1$, se ejecuta la línea 2 y 3.
\begin{eqnarray*}
T(i,j) & \geq & c_12^{max(1,1)} \\
T(i,j) & \geq & 2^1 
\end{eqnarray*}
\item Paso inductivo:
\begin{eqnarray*}
\Omega(2^{max(i,j)}) = T(i,j) \hspace{1cm} 1 \leq i \leq m-1 \hspace{0.5cm} \wedge \hspace{0.5cm} 1 \leq j \leq n-1 \\
\end{eqnarray*}
Se sabe que:
\begin{eqnarray*}
T(m,n) &=& \sum^2_{k=m}T(k-1,n-1)+\sum^{2}_{k=n}T(m-1,k-1) \\
T(m,n) &=& \sum^2_{k=m}T(k-1,n-1)+\sum^{2}_{k=n}T(m-1,k-1) \geq \sum^2_{k=m}T(k-1,n-1) \\
T(m,n) &\geq& \sum^2_{k=m}T(k-1,n-1) \\
T(m,n) &\geq& T(m-1,n-1) \\
T(m,n) &\geq& 2^{max(m-1,n-1)} \\
T(m,n) &\geq& 2^{max(m,n)-1} \\
T(m,n) &\geq& \frac{1}{2}2^{max(m,n)}
\end{eqnarray*}
Se concluye que $T(m,n)=\Omega(2^{max(m,n)})$
\begin{eqnarray*}
T(m,n) &\geq& \frac{1}{2}2^{max(m,n)} \hspace{1cm} \text{ m $\geq$ 1 $\wedge$ n $\geq$ 1}
\end{eqnarray*}

\end{itemize}

\section{Algoritmo Memoizado}

\subsection{Pseudocódigo}
Se reutiliza las funciones MatchGroup y MatchDivision definidos en la anterior sección. Antes de implementar el algoritmo, se inicializa toda la matriz en cero.

\newpage
\begin{algorithm}
\caption{Devuelve el min matching utilizando una matriz como apoyo}
\begin{algorithmic}[1]
\Procedure{Opt}{$i,j$}
    \If{Matrix[i][j] != 0}
    \State{ \Return Matrix[i][j]}
    \EndIf
    \If{$i==1$}
    \State{ Matrix[i][j] = MatchDivision(i, 0, j) }
    \State{ \Return Matrix[i][j] }
    
    \ElsIf{$j==1$}
    \State{ Matrix[i][j] = MatchGroup(0, i, j) }
    \State{ \Return Matrix[i][j] }
    
    \Else
    \State{Tuplas min\_resultk.weight = $\infty$}
    \State{Tuplas min\_resultl.weight = $\infty$}
    \For{k = i \textbf{down to} 2}
    \State{Match = MATCHGROUP( $k, i, j$ )}
    \State{SubProblem = Opt( $k-1, j-1$ )}
    \State{result = SubProblem + Match }
    \If{min\_resultk.weight $>$ result.weight}
    \State{min\_resultk = result}
    \EndIf
    \EndFor
    
    \For{l = j \textbf{down to} 2}
    \State{Match = MATCHDIVISION( $i, l, j$ )}
    \State{SubProblem = Opt( $i-1, l-1$ )}
    \State{result = SubProblem + Match }
    \If{min\_resultl.weight $>$ result.weight}
    \State{min\_resultl = result}
    \EndIf
    \EndFor
    
    \State{Matrix[i][j] = min( min\_resultl, min\_resultk )}
    \State{\Return Matrix[i][j]}
    
\EndProcedure
\end{algorithmic}
\end{algorithm}

\subsection{Tiempo de ejecución}

El tiempo de ejecucion de este algoritmo está dado por:

\begin{equation}
T(i,j) = ( \text{ \# SubProblemas } ) * ( \text{Tiempo por SubProblema} )  
\end{equation}

Se esta tomando en consideracion que las funciones MatchDivision y MatchGroup se ejecutan en tiempo constante, por lo que solo se necesita contabilizar el numero de subproblemas existentes$^1$. Se replantea la ecuación (1):
\begin{equation}
T(i,j) = ( \text{ \# SubProblemas } ) * c
\end{equation}  

Asimismo, debido a las llamadas recursivas de los subproblemas $T(k-1,j-1)$ y $T(i-1,l-1)$ en las lineas 15 y 21 del algoritmo memoizado la cantidad de subproblemas que se resuelven son:

$$(\text{\# SubProblemas}) = (i-1)*(j-1)+\underbrace{1}_{\text{Problema original: } T(i,j)}$$

Entonces, para $T(m,n)$ se obtiene:
\begin{eqnarray*}
T(m,n) &=& (m-1)*(n-1)+1 \\
m*n &\geq& (m-1)*(n-1)+1 = T(m,n) \\
m*n &\geq& T(m,n)
\end{eqnarray*}

Por lo que se demuestra que $T(m,n)=O(mn)$

 
\section{Algoritmo Dinámico}
\subsection{Pseudocódigo}
En primer lugar, se define la función OPT\_Result($i, j$). 
\newpage
\begin{algorithm}
\caption{Devuelve el min matching utilizando una matriz como apoyo}
\begin{algorithmic}[1]
\Procedure{OPT\_Result}{$i,j$}
    \If{i == 1}
        \State{Matrix[i][j] = GetMatchDivision(i, 1, j)}
    \ElsIf{j == 1}
        \State{Matrix[i][j] = GetMatchGroup(1, i, j)}
    \Else
        \State{min\_resultk = math.inf}
        \State{min\_resultl = math.inf}
    
        \State{indexMinGroup = 1}
        \State{indexMinDivision = 1}
    
        \For{k = i \textbf{down to} 1}
                \State{Match = GetMatchGroup(k, i, j)}
                \State{SubProblem = Matrix[k-1][j-1]}
                \State{result = SubProblem + Match}
                
                \If{min\_resultk $>$ result}
                    \State{min\_resultk = result}
                    \State{indexMinGroup = k}
                \EndIf
        \EndFor
        \For{l = j \textbf{down to} 1}
                \State{ Match = GetMatchDivision(i, l, j)}
                \State{SubProblem = Matrix[i-1][l-1]}
                \State{result = SubProblem + Match}
                
                \If{min\_resultl $>$ result}
                    \State{min\_resultl = result}
                    \State{indexMinDivision = l}
                \EndIf
        \EndFor
         \If{min\_resultl $>$ min\_resultk}
            \State{Matrix[i][j] = min\_resultk}
            \State{minSubProblem[i, j] = (indexMinGroup-1, j-1)}
        \Else
            \State{Matrix[i][j] = min\_resultl}
            \State{minSubProblem[i, j] = (i-1, indexMinDivision-1)}
        \EndIf
    \EndIf
\end{algorithmic}
\end{algorithm}

Finalmente, se diseña el algoritmo de programacion dinámica. 

\begin{algorithm}
\caption{Devuelve min matchings usando DP}
\begin{algorithmic}[1]
\Procedure{DynamicProgramming}{$x,y$}
     \For{i = 1 \textbf{to} x}
         \For{j = 1 \textbf{to} y}
            \State{OPT\_Result(i, j)}
         \EndFor
     \EndFor
    \State{OPT\_Result(x, y)}
    \State{\Return{Matrix[x][y]}
\end{algorithmic}
\end{algorithm}
\newpage
\subsection{Tiempo de Ejecución}
Para demostrar la complejidad del algoritmo, se observa lo siguiente:
\begin{itemize}
    \item En la función OPT\_RESULT($m, n$), se observan dos bucles independientes:
    \begin{eqnarray*}
    T(m,n) &=& m + n \\
    T(m,n) &\leq& max\{m,n\} \\
    O(max\{m, n\}) &=& T(m, n)
    \end{eqnarray*}
    \item En la función DynamicProgramming($x,y$), se observan dos bucles anidados llamando a la función OPT\_RESULT($m, n$), por lo que: \begin{eqnarray*}
    T(m,n) &=& (m-1)(n-1)O(max\{m,n\}) \\
    T(m,n) &\leq& (m)(n)O(max\{m,n\}) \\
    T(m,n) &\leq& max\{m,n\}^2O(max\{m,n\}) \\
    T(m,n) &\leq& max\{m,n\}^3 \\
    O(max\{m, n\}^3) &=& T(m, n)
    \end{eqnarray*}
\end{itemize}
Por lo tanto el tiempo de ejecución del algoritmo de programación dinámica es $O(max\{m,n\}^3)$
 
\section{Algoritmo Dinamico Mejorado}
\subsection{Pseudocódigo}
Se implementa la siguiente función para calcular el $u$

\begin{algorithm}
\caption{Inicializar la constante u}
\begin{algorithmic}[1]
\Procedure{InicializarU}{}
    \State{u = sumaBloquesA[len(A)-1]/sumaBloquesB[len(B)-1]}
\end{algorithmic}
\end{algorithm}

Las listas \textbf{sumaBloquesA} y \textbf{sumaBloquesB} obtienes las sumas acumuladas de los pesos de los bloques de \textbf{A} y \textbf{B}.

Luego, se realizan los siguientes cambios en las funciones de GetMatchDivision($i, m, n$) y GetMatchGroup($r, s, j$). Estas funciones devuelven los pesos de los matchs.

\begin{algorithm}
\caption{Obtener el peso de un match de división}
\begin{algorithmic}[1]
\Procedure{GetMatchDivision}{$i, m, n$}
    \If{m == n}
        \State{\Return abs(A[i].longitud/B[m].longitud - u)}
    \EndIf
    \If{m == 0}
        \State{\Return abs(A[i].longitud/sumaBloquesB[n] - u)}
    \EndIf
    \State{suma = sumaBloquesB[n] - sumaBloquesB[m-1]}
    \State{\Return abs(A[i].longitud/suma - u)}
\end{algorithmic}
\end{algorithm}

\begin{algorithm}
\caption{Obtener el peso de un match de agrupacion}
\begin{algorithmic}[1]
\Procedure{GetMatchGroup}{$r, s, j$}
    \If{r == s}
        \State{\Return abs(A[r].longitud/B[j].longitud - u)}
    \EndIf
    \If{r == 0}
        \State{\Return abs(sumaBloquesA[s]/B[j].longitud - u)}
    \EndIf
    \State{suma = sumaBloquesA[s] - sumaBloquesA[r-1]}
    \State{\Return abs(suma/B[j].longitud - u)}
\end{algorithmic}
\end{algorithm}

Se realizan esos cambios para el correcto funcionamiento del algoritmo de programación dinámica mejorada.

\subsection{Tiempo de Ejecución}
Como los cambios realizados no influyen en la notación de O-grande, el tiempo de ejecución es el mismo que el algoritmo de programación dinámica.

\begin{eqnarray*}
    O(max\{m, n\}^3) &=& T(m, n)
\end{eqnarray*}

\section{Algoritmos de Transformaciones de Matrices}
Para la transformación de matrices se desarrollo un algoritmo general para los tres métodos de transformación de matrices (Greedy , Dinamica, DinamicaMejorada).
Dentro de este algoritmo se encuentra la función MIN\_MATCHING representa los algoritmos anteriormente mencionados, que podran ser greedy, dinamica y dinamica mejorada. 
\subsection{Pseudocódigo}
\begin{algorithm}
\caption{Devuelve un conjunto de matches}
\begin{algorithmic}[1]
\Procedure{Transformacion\_MIN}{matrixA, matrixB, GetSubmtachings = False}
    \If{GetSubmatching}
        \For{i = 1 \textbf{to} size(matrixA)}
            
            \State{result = MIN\_MATCHING(matrixA[i] , matrixB[i] , GetSubmatching)}
            \State{MatrixMatchings.insert(result)}
        \EndFor
    \Else
        \State sumatoria = 0.0
         \For{i = 1 \textbf{to} size(matrixA)}
            \State{restult = Greedy.MIN\_MATCHING(matrixA[i],matrixB[i])}
            \State{sumatoria = sumatoria + result}
            \State{MatrixMatchings.insert(matchings)}
         \EndFor
         \State{\Return{sumatoria}}
    \EndIf
\end{algorithmic}
\end{algorithm}
\newpage
\subsection{Tiempo de Ejecucion}
A cada algoritmo descrito anteriormente se debe agregar el número de filas
\begin{itemize}
    \item \textbf{Transformacion Greedy}, dos matrices A y B de ceros y unos de tamaño p x q. 
    \begin{eqnarray*}
        O(pq) &=& Tr(m, n)
    \end{eqnarray*}
    \item \textbf{Transformacion Programcion Dinamica}, dos matrices A y B de ceros y unos de tamaño p x q.
    \begin{eqnarray*}
        O(pq^3) &=& Tr(m, n)
    \end{eqnarray*}
    \item \textbf{Transformacion Programcion Dinamica Mejorada}, dos matrices A y B de ceros y unos de tamaño p x q.
    \begin{eqnarray*}
        O(pq^3) &=& Tr(m, n)
    \end{eqnarray*}
\end{itemize}
\newpage
\section{Procesamiento de imagenes}
Para esta sección del proyecto que fue realizado en python, se utilizó la libreria Pillow, para facilitar la manipulación de imagenes.

\subsection{Escala de Grises}
Para convertir la matriz de pixeles RGB a Escala de grises, se han implementado 4 funciones, $LUM\_601$, $LUM\_709$,$LUM\_240$ y $LUM\_input$. Este ultimo permite setear los factores de conversion a los que el usuario le pasa. Todas las anteriores funciones llaman a la siguiente subrutina convert:

\begin{lstlisting}
def convert(image, R, G, B):
    ConvertedMatrix01 = []

    for y in range(height):
        array01 = []
        for x in range(width):
        
            RGB = image.getpixel((x,y))
            Gris = int(RGB[0]*R + RGB[1]*G + RGB[2]*B)
            if(Gris > 122):
                array01.append(1)
            else:
                array01.append(0)
        ConvertedMatrix01.append(array01)
    return ConvertedMatrix01
\end{lstlisting}

El umbral utilizado para generar los arreglos de bloques ha sido 122. El cual si el Valor de gris es menor que ese valor se agregará como 1, caso contrario, será 0.

\section{Animación de Imagenes}
\subsection{Terminos importantes}
Para la animacion de imagenes, estamos identificando 3 conjuntos de pixeles:
\begin{itemize}
    \item Submatchings: relacion entre un bloque a varios (division o agrupacion)  
    \item Antimatchings: Todos los bloques de 0's a los lados de los Submatchings, los cuales no son afectados por estos mismos
    \item pixelesLibres: Todos los pixeles que no sean afectados por los dos anteriores( podria ser que existan en el medio de dos bloques divididos o que el matching este justo al final, por lo que los bloques superiores o inferiores deberian desaparecer o aparecer en la transformacion.

\subsection{Pseudocodigo}
    Pasando con el algoritmo:
\end{itemize}

\begin{algorithm}
\caption{Genera una lista de K imagenes intermedias }
\begin{algorithmic}[1]
\Procedure{GenerarImagenesIntermedias}{MatrixMatchings,img1,img2,directorio}:
    \State {MEGA\_MATRIX = []}
    \For{i = NUM\_IMG+1 \textbf{down to} 1:}
        \State{listaVacia =[]}
        \State{Mega\_Matrix.append(listaVacia)}
    \EndFor
    \For{i in range(len(MatrixMatchings) )}
        \State{row11 = GetRow(img1,i}
        \State{row12 = GetRow(img2,i}
        \State{matchings = MatrixMatchings[i]}

        \State{antimatchigs = us.GetAntiMatching(matchings)}
        \For{submatching in matchings: }
            \State submatching.getProporcionalidad()
        \EndFor
        \For{submatching in antimatchings: }
            \State submatching.getProporcionalidad()
        \EndFor
        \State{matrix = us.generarMatrizPorlinea(matchings, antimatchings,row11,row12}
        \For{j in range(len(MEGA\_MATRIX)):}
            \State MEGA\_MATRIX[j].append(matrix[j])
        \EndFor
    \For{ i in range(NUM\_IMG+1):}
        \State{ pil.ArmarImagen(img2,MEGA\_MATRIX,i,directorio)}
    \EndFor

\end{algorithmic}
\end{algorithm}

La funcion us.generarMatrizPorLinea, recibe los matchings y antimatchings, y las filas, para poder ejecutar la logica de la animacion y devuelve una matriz de NUM\_IMG+1 * ancho de la imagen. 

Por otra parte, la función ArmarImagen recibe todas las matrices anteriormente generadas y distribuye los pixeles para poder generar las N

\section{Repositorio}
Enlace al repositorio git:
https://github.com/cesar214567/ProyectoADA 




\section{Bibliography}
\begin{enumerate}
    \item Demaine,Erik.(2011).Lecture 19: Dynamic Programming I: Fibonacci, Shortest Paths.Dynamic Programing. United States. MIT
\end{enumerate}

\end{document}